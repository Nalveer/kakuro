\documentclass[12pt]{article}

\pagestyle{empty}
\setcounter{secnumdepth}{0}

\topmargin=0cm
\oddsidemargin=0cm
\textheight=22.0cm
\textwidth=16cm
\parindent=0cm
\parskip=0.15cm
\topskip=0truecm
\raggedbottom
\abovedisplayskip=3mm
\belowdisplayskip=3mm
\abovedisplayshortskip=0mm
\belowdisplayshortskip=2mm
\normalbaselineskip=12pt
\normalbaselines

\begin{document}

\vspace*{0.2in}
\centerline{\bf\Large Diary}

\vspace*{0.2in}
\centerline{\bf\Large Name: Tiffany Ah King  Student ID: 40082976}

\vspace*{0.2in}
\centerline{\bf\Large Team PK-A}

\vspace*{0.2in}
\centerline{\bf\Large 09 February 2020}

\section{Iteration 1}

{\bf Date:} 09 January 2020\\
{\bf Start Time:} 20:15\\
{\bf End Time:} 21:00 \\
{\bf Who:} Tiffany Ah King, Isabelle Charette, Brian Gamboc-Javiniar, Vsevolod Ivanov,
Chang Liu, Nolan Mckay, Nalveer Moocheet, Hoang Thuan Pham, Audrey-Laure St-Louis, Jia Ming Wei\\
{\bf Where:} H-920 \\
{\bf Activities:} We met and introduced each other, learned our interests and motivations. We talked about what are expected from everyone which are honesty, respect and hardwork Then, we determined and create platforms for communications. We took note of everyone's emails.\\
{\bf Outcomes:} For communnication, we have chosen Discord which I created a server and added everyone and for programming it was GitHub. We fixed the next meeting at the lab Wednesday. Jia Ming created the Trello, Vsevold created the GitHub repository \\ \\


{\bf Date:} 15 January 2020\\
{\bf Start Time:} 7.15 p.m\\
{\bf End Time:} 9.00 p.m\\
{\bf Who:}  Tiffany Ah King, Isabelle Charette, Brian Gamboc-Javiniar, Vsevolod Ivanov,
Chang Liu, Nolan Mckay, Nalveer Moocheet, Hoang Thuan Pham, Audrey-Laure St-Louis, Jia Ming Wei \\
{\bf Where:} H831 \\
{\bf Activities:} The tutor, Mohammad Reza Rejali, covered the "programming cycle" - Planning, Analysis, Design, Implementation and Maintenance. He introduced us to the difference of the lecture, tutorial and lab and their importance. He talked about the goal of a software, its problems, its solutions and its benefits. The tutor explained and went over the project description, about the demo and the diffent roles (Coder, Documenter, Organizer).\\

Then, team PK-A discussed about who will do what for Iteration. There will be 4 coders, 5 documenters and 1 organizer. \\

Coders: Audrey-Laure St-Louis, Isabelle Charette, Nalveer Moocheet and Vsevolod Ivanov\\
Documenters: Tiffany Ah King, Chang Liu, Brian Gamboc-Javiniar, Nolan Mckay and  Hoang Thuan Pham\\

Organizer: Jia Ming Wei\\

As a team, we discussed about the different user cases and we came up with five. Each coder and documenter have each one user case to take care of for Iteration 1. 
\\

{\bf Outcomes:} Everyone in the group got a role and we are aware of the test cases.We all have specific but flexible tasks to work / explore more before the next meeting. We started to think in terms of MVC for the architecture of the code.We exchange schedules through Discord\\\\

{\bf Date:} 16 January 2020\\
{\bf Start Time:} 20:30\\
{\bf End Time:} 21:20\\
{\bf Who:} Tiffany Ah King, Brian Gamboc-Javiniar,\\
 Hoang Thuan Pham, Jia Ming Wei\\
{\bf Where:} H-523\\
{\bf Activities:} During the tutorial, the TA about the user stories and user diagram and gave us an example of the bank.\\
{\bf Outcomes:}In group, we came up with different user cases and we agreed to share our findings on Discord.\\ \\


{\bf Date:} 22 January 2020\\
{\bf Start Time:} 14h00\\
{\bf End Time:} 16h00\\
{\bf Who:} Tiffany Ah King\\
{\bf Where:} Home\\
{\bf Activities:} I read and learned more about LaTex through Overleaf and the sample Latex template for the use case document. I skimmed over the "Monopoly Project Analysis and Development Plan Version 1.3" LaTex pdf that the professor, Greg Butler posted on his site. It was very resourceful and it helped me to have a headstart on the documentation. Futhermore, I documented myself on the Kakuro game and went on several websites like wikipedia among others.\\
{\bf Outcomes:} I wrote the title cover, the table of content, the introduction and try to make it have a good alignment.I had a problem with the date appearing at the top each time and I was able to resolve it. I used the LaTeX Base online to start with.\\ \\


{\bf Date:} 22 January 2020\\
{\bf Start Time:} 19:15\\
{\bf End Time:} 20:30\\
{\bf Who:} Tiffany Ah King, Isabelle Charette, Brian Gamboc-Javiniar, Vsevolod Ivanov,\\
Chang Liu, Nolan Mckay, Nalveer Moocheet, Hoang Thuan Pham, Audrey-Laure St-Louis, Jia Ming Wei\\
{\bf Where:} H-831\\
{\bf Activities:} We updated our work on where we were at that moment. The documenters were supposed to learned LaTex and to review  the user cases. Everyone tried to clear some confusion and misunderstandings and shed light on the implementation details. We also discussed about the upcomming demo lab of the Proof of Concept (PoC) which will be held on Wednesday 29 January in H831 at 19h15.\\
{\bf Outcomes:} Everything got more clear for everyone and I tried to motivate everyone and tell them not to stress. The coders and documenters have their own discussion on their parts. The documentation group including me decided who will do what for the "Requirement Document" for Iteration1 . I am in charge of the title page, the introduction, the table of content, the references and the game mechanics . We agreed to do a voice call over Discord on Sunday 26 January at 19h00.\\ \\ 


{\bf Date:} 26 January 2020\\
{\bf Start Time:} 19:00\\
{\bf End Time:} 19:45\\
{\bf Who:} Tiffany Ah King, Brian Gamboc-Javiniar,
Chang Liu, Nolan Mckay, Hoang Thuan Pham, Jia Ming Wei\\
{\bf Where:} Home via Discord Voice call\\
{\bf Activities:} We cleared any confusion and answered to each other. We talked about the demo on Wednesday 29th and we streamed on some user case diagram.\\
{\bf Outcomes:} Everyone is on the same page now and we decided to set Tuesday as a day to merge all of our documents. Everyone knows that they need to finish theri part for the demo to show the TA\\


%\section{Iteration 2}

%\section{Iteration 3}

\end{document}