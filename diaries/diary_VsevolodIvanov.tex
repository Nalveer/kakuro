\documentclass[12pt]{article}

\pagestyle{empty}
\setcounter{secnumdepth}{0}

\topmargin=0cm
\oddsidemargin=0cm
\textheight=22.0cm
\textwidth=16cm
\parindent=0cm
\parskip=0.15cm
\topskip=0truecm
\raggedbottom
\abovedisplayskip=3mm
\belowdisplayskip=3mm
\abovedisplayshortskip=0mm
\belowdisplayshortskip=2mm
\normalbaselineskip=12pt
\normalbaselines

\begin{document}

\vspace*{0.2in}
\centerline{\bf\Large Diary}

\vspace*{0.2in}
\centerline{\bf\Large Name: Vsevolod Ivanov   Student ID: 40004286}

\vspace*{0.2in}
\centerline{\bf\Large Team PK-A}

\vspace*{0.2in}
\centerline{\bf\Large 09 January 2020}

\section{Iteration 1}

{\bf Date:} 09 January 2020\\
{\bf Start Time:} 20:15\\
{\bf End Time:} 21:00 \\
{\bf Who:} Tiffany Ah King, Isabelle Charette, Brian Gamboc-Javiniar, Vsevolod Ivanov,\\
Chang Liu, Nolan Mckay, Nalveer Moocheet, Hoang Thuan Pham, Audrey-Laure St-Louis, Jia Ming Wei\\
{\bf Where:} H-920 \\
{\bf Activities:} We met eachother, learned our interests and motivations. Then, we determined and create platforms for communications, kanban and code. We took note of everyone's emails.\\
{\bf Outcomes:} For communnications, we have chosen Discord, for Kanban and User stories Trello or maybe Github and for programming it was GitHub. We fixed the next meeting at the lab Wednesday. Teammates created the Trello and Dicord projects and I created the GitHub repository.\\

{\bf Date:} 15 January 2020\\
{\bf Start Time:} 19:15\\
{\bf End Time:} 22:00\\
{\bf Who:} Tiffany Ah King, Isabelle Charette, Brian Gamboc-Javiniar, Vsevolod Ivanov,\\
Chang Liu, Nolan Mckay, Nalveer Moocheet, Hoang Thuan Pham, Audrey-Laure St-Louis, Jia Ming Wei\\
{\bf Where:} H-831\\
{\bf Activities:} We discovered the project, learned about the roles and iterations. We determined our roles for organiser (1) coders (4) and rest in the documentation! We wrote first user stories brief and assigned coders and documenters for each one to dig deeper into.\\
{\bf Outcomes:} We all have specific but flexible tasks to work / explore more before the next meeting. We started to think in terms of MVC for the architecture of the code. Later that night, I pushed a basic MVC class template for the Game.\\\\

{\bf Date:} 16 January 2020\\
{\bf Start Time:} 20:15\\
{\bf End Time:} 20:30\\
{\bf Who:} Tiffany Ah King, Isabelle Charette, Brian Gamboc-Javiniar, Vsevolod Ivanov,\\
Chang Liu, Nolan Mckay, Nalveer Moocheet, Hoang Thuan Pham, Audrey-Laure St-Louis, Jia Ming Wei\\
{\bf Where:} H-920\\
{\bf Activities:} We ensured that everyone is fine with their tasks.\\
{\bf Outcomes:} The organizator for this iteration assigned me a second role of organiznig the coders team members.\\

{\bf Date:} 17 January 2020\\
{\bf Start Time:} 10:40\\
{\bf End Time:} 13:00\\
{\bf Who:} Isabelle Charette, Vsevolod Ivanov, Nalveer Moocheet, Audrey-Laure St-Louis\\
{\bf Where:} Lb-353(Kenya)\\
{\bf Activities:} We met between coders to jump start the whole project for everyone. We started by deconstructing the project into game requirements, domain models and use cases. In order to be all aligned at the same goal with our previous roles, we draw on a white board and put into draw.io importable format the following diagrams: Domain Model: Game Management, MVC of the Game with Cell and Player extensions and the Player Use Cases. Then, we assigned them for Iteration 1 and 2 according to must have and the relation of our assigned roles.\\
{\bf Outcomes:} We shared it with rest of the team on Discord and they had a very good reception of our diagrams. Finally, it was more clear the parts that we have to code. We aim at starting to put them together along with the documentation done by its team during the next lab.\\

\pagebreak

{\bf Date:} 22 January 2020\\
{\bf Start Time:} 19:15\\
{\bf End Time:} 20:30\\
{\bf Who:} Tiffany Ah King, Isabelle Charette, Brian Gamboc-Javiniar, Vsevolod Ivanov,\\
Chang Liu, Nolan Mckay, Nalveer Moocheet, Hoang Thuan Pham, Audrey-Laure St-Louis, Jia Ming Wei\\
{\bf Where:} H-831\\
{\bf Activities:} We updated the state of things for everyone, cleared some misunderstandings and shed light on the implementation details. We discussed the upcomming demo lab of the Proof of Concept (PoC).\\
{\bf Outcomes:} Everything got more clear and everyone jumped on their parts. I setup GitHub Actions with Ant to launch JUnit tests after each commit is pushed which will allow us to avoid regression and ensure that code we push works and it encourages TDD (Test-Driven-Development) increasing our code coverage. I must note that I'm a bit worried because I haven't seen the docs (just an outline), the codes haven't been pushed yet besides mine. Just in case, I will implement a console UI gameplay as a fallback for the PoC. The latter will reduce time on some less important parts like a complete and random board generation.\\

%\section{Iteration 2}

%\section{Iteration 3}

\end{document}
